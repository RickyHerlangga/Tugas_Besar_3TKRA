\documentclass[conference]{IEEEtran}

\usepackage{cite}
\usepackage{graphicx}

\title{An Improvement of Weighted PageRank to Handle the Zero Link Similarity }
\author{\IEEEauthorblockN{Ricky Herlangga\\ 2022018}\\
\IEEEauthorblockA{\textit{Fakultas Teknologi Informasi}\\
\textit{Institut Teknologi Batam}\\
Batam, Indonesia\\
2022018@student.iteba.ac.id}}

\usepackage

%START DOCUMENT
\begin{document}

% Judul
\maketitle

%ABSTRAK
\begin{abstract}
Algoritma PageRank yang terkenal memanfaatkan struktur tautan untuk menghitung peringkat kualitas halaman.
Ini pada dasarnya memberikan jumlah probabilitas yang sama ke halaman tetangga pada sebuah halaman.
Sebagai ekstensinya, algoritma PageRank berbobot telah diusulkan yang memberikan bobot berbeda pada tautan keluar dari sebuah halaman.
Beberapa algoritma PageRank berbobot menggunakan kesamaan antar halaman sebagai bobot.
Di halaman web Korea, kami menemukan bahwa terkadang memiliki nilai nol untuk kesamaan antar halaman dari halaman tetangga karena karakteristik bahasa.
Makalah ini mengusulkan algoritma PageRank berbobot yang ditingkatkan yang dapat menangani kesamaan antar-halaman nol seperti itu.
Metode yang diusulkan telah diimplementasikan menggunakan paradigma MapReduce untuk penanganan data besar, dan telah dievaluasi melalui halaman web Wikipedia bahasa Korea dan dibandingkan dengan dua metode lainnya.
\end{abstract}

%kata kunci
\begin{IEEEkeywords}
    Weighted     PageRank;     Similarity;     MapReduce; TFIDF 
\end{IEEEkeywords}

%Pendahuluan
\section{introduction}
Saat ini, ketika orang ingin mengetahui sesuatu, banyak dari mereka mencoba mencarinya di Internet.
Mereka akan kewalahan jika terlalu banyak halaman yang diberikan sebagai halaman yang relevan dalam pencarian web.
Dalam pencarian informasi (IR), peringkat telah menjadi salah satu isu penting.
To  sort  out  influential  ones  among  searched  pages,  various ranking algorithms have been proposed. ~\cite{brin1998anatomy,xing2004weighted,qiao2010simrank}\\

PageRank\cite{brin1998anatomy} adalah salah satu algoritma peringkat terkenal yang menggunakan struktur tautan Web.
ini mengasumsikan bahwa seorang peselancar berjalan secara acak di atas halaman web dan mencoba menentukan distribusi statis peselancar.
Dengan metafora penderitaan acak, semakin banyak tautan yang dimiliki halaman, semakin tinggi peringkatnya.
Di PageRank, penderitaan membuat jalan acak ke halaman tetangga dengan probabilitas yang sama.
Kadang-kadang probabilitas yang sama ini tampaknya tidak masuk akal karena beberapa tautan terhubung ke halaman tetangga yang jauh lebih penting.\\

Untuk mengatasi situasi ini, algoritma PageRank berbobot~\cite{qiao2010simrank,xing2004weighted} telah di usulkan.
Mereka memperhitungkan baik distribusi jumlah in-link untuk node tetangga, jumlah kunjungan ke halaman tetangga, atau kesamaan antar halaman.
Masing-masing memiliki pro dan kontra. Pembobotan berbasis kesamaan antar halaman terdengar bagus untuk peringkat berbasis konten.\\

Kami telah mencoba algoritma PageRank berbobot berbasis kesamaan antar halaman ke halaman Wikipedia bahasa Korea. Untuk menghitung kesamaan antar halaman, kami menggunakan model vektor~\cite{}
Untuk mendapatkan representasi vektor untuk halaman, pertama-tama kami melakukan analisis morfologi untuk mengekstrak kata-kata.
Berbeda dengan bahasa barat, kata benda dalam bahasa Korea menyampaikan informasi yang paling berarti.
Karena karakteristik bahasa, kata benda diekstraksi untuk mengidentifikasi kata kunci.
Kata kunci diidentifikasi menggunakan istilah frekuensi dan informasi frekuensi dokumen terbalik. Dengan kata kunci, representasi vektor untuk halaman diperoleh.
Dengan kata kunci, representasi vektor untuk halaman diperoleh. Kemudian, kebetulan memiliki kesamaan nol ketika kesamaan antar halaman dihitung menggunakan produk dalam dari vektor-vektor tersebut.
Sangat canggung untuk halaman tetangga yang tidak memiliki kesamaan.
makalahnya berkaitan dengan peningkatan pada PageRank berbobot berbasis kesamaan antar halaman untuk menangani kasus-kasus dengan tautan nol kesamaan.\\

Sisa makalah ini disusun sebagai berikut: Bagian II menyajikan PageRank dan variannya secara lebih rinci.
Bagian III memperkenalkan peningkatan pada PageRank berbobot, dan Bagian IV menunjukkan beberapa hasil eksperimen untuk metode yang diusulkan.
Kami menarik kesimpulan di Bagian V.

\section{Related Works}
\subsection{PageRank Algoritm}
PageRank\cite{brin1998anatomy} adalah algoritma perbatasan yang memberi peringkat halaman dengan mengacu pada struktur tautan Web.
PageRank memperlakukan halaman sebagai node dan hyperlink sebagai tepi grafik.
Setiap node memiliki nilai Rank sendiri dan mendistribusikannya secara merata ke tetangganya.
Distribusi berulang tanpa batas sampai semua nilai peringkat konvergen.
Distribusi stasioner dari Peringkat dianggap sebagai skor Peringkat akhir halaman.
Untuk mencegah peningkatan nilai Peringkat tanpa batas, jumlah semua Peringkat dibatasi menjadi 1, dan juga untuk setiap nilai Peringkat tidak lebih besar dari 1.
Nilai peringkat rj dari node j dihitung sebagai berikut:

\begin{equation}
    r_j = \sum_{i \rightarrow j} \frac{r_j}{L_{out}(i)}
\end{equation}

\begin{equation}
    \sum r_j = 1
\end{equation}

dimana Lout adalah jumlah out-link dari node i.Setiap
simpul i mendistribusikan skor peringkatnya ri secara merata
simpul tetangganya j. SEBUAH node j mengumpulkan semua
skor Rank yang dikirimkan dari tetangga node dan ambil jumlah mereka sebagai skor Peringkat baru rj . Peringkat ini proses
distribusi dapat dinyatakan dengan kedekatan stokastik matriks
M dan vektor pangkat r. Matriks M adalah tetangganya
matriks untuk web yang mengkodekan hubungan lingkungan
antara halaman dan distribusi stasioner nilai Peringkat. Baru
Nilai peringkat r
(t+1) dihitung sebagai berikut:

\begin{equation}
    r(i+l)=Mr(t)
\end{equation}
\subsection{WeightedPageRank Algoritm}
Surfers sebenarnya tidak melakukan random walk seperti
di PageRank. Untuk mengakomodasi karakteristik perilaku
seperti itu, pembobotan Algoritma PageRank telah diusulkan
yang memungkinkan penderitaan untuk membuat transisi
probabilistik yang tidak merata ke tetangga halaman. [2], [3],
[8]


\subsubsection{ Weighted  PageRank  based  on  the  number  of  in-links  of  neighboring pages}
Xing dan Ghorbani [2] mengusulkan PageRank algoritma
yang memberikan lebih banyak porsi Rank ke tetangga
halaman dengan lebih banyak tautan. Ya tidak cukup
mencerminkan perilaku peselancar yang sebenarnya, karena
hanya informasi struktur topologi digunakan.

\subsubsection{Weighted PageRank based on Similarity Measure}
Qiaoet al. [3] menyarankan varian PageRank berbobot
algoritma, yang disebut SimRank, yang mendistribusikan nilai
Rank di porsi kesamaan antar halaman. Untuk menerapkan
metode, semua kesamaan halaman berpasangan perlu dihitung
lebih awal. Secara komputasi mahal untuk menerapkan
metode ini untuk skala besar volume halaman. Oleh karena
itu untuk menerapkan metode, perlu infrastruktur komputasi
paralel terdistribusi seperti Hadoop Pengurangan Peta [14].

\subsubsection{Weighted PageRank based on visits of links}
Qiaoet al. [3] menyarankan varian PageRank berbobot
algoritma, yang disebut SimRank, yang mendistribusikan nilai
Rank di porsi kesamaan antar halaman. Untuk menerapkan
metode, semua kesamaan halaman berpasangan perlu dihitung
lebih awal. Secara komputasi mahal untuk menerapkan
metode ini untuk skala besar volume halaman. Oleh karena
itu untuk menerapkan metode, perlu infrastruktur komputasi
paralel terdistribusi seperti Hadoop Pengurangan Peta [14].

\subsection{Hub and Authorities Algorithm }
Qiaoet al. [3] menyarankan varian PageRank berbobot
algoritma, yang disebut SimRank, yang mendistribusikan nilai
Rank di porsi kesamaan antar halaman. Untuk menerapkan
metode, semua kesamaan halaman berpasangan perlu dihitung
lebih awal. Secara komputasi mahal untuk menerapkan
metode ini untuk skala besar volume halaman. Oleh karena
itu untuk menerapkan metode, perlu infrastruktur komputasi
paralel terdistribusi seperti Hadoop Pengurangan Peta [14].

\subsection{Distributed and Parallel Computing }
Pengambilan informasi dari repositori data besar, seperti
Web dan penyimpanan data besar membutuhkan infrastruktur
komputasi yang menyimpan dan memproses data tersebut.
Kita dapat menggunakan salah satu dari sistem superkomputer
atau komputasi terdistribusi dan paralel sistem\\

Hadoop [14] adalah infrastruktur komputasi yang baik
yang dapat ditetapkan dalam biaya ekonomi. Ini adalah
proyek Apache untuk platform komputasi terdistribusi yang
menyediakan seperti sistem file terdistribusi yang disebut
HDFS (Hadoop Distributed File System) dan kerangka kerja
komputasi paralel terdistribusi yang disebut Kurangi Peta.
Kerangka kerja MapReduce mengatur pekerjaan ke dalam Peta
tugas dan Mengurangi tugas. Data input dipartisi dan diproses
oleh proses Peta, dan hasil pemrosesannya dibentuk menjadi pasangan nilai kunci. Hasil tugas peta dikocok menjadi
Reduce tugas sesuai dengan kunci mereka. Kurangi proses
agregat nilai dengan kunci yang sama, untuk mendapatkan
hasil akhir. Ini kerangka kerja komputasi memungkinkan kita
untuk menangani beban berat komputasi seperti komputasi
kesamaan halaman berpasangan.


%proposed algorithm
\section{THE PROPOSED ALGORITHM}
PageRank berbobot berbasis kesamaan antar halaman
pada kesamaan tidak dapat menangani situasi yang antar
halaman kesamaannya adalah 0. Untuk menghadapi situasi
ini, kami mengusulkan a metode untuk menjaga kesamaan
antar halaman nol dan untuk menyesuaikan bobot untuk
distribusi nilai Rank.
Ekstraksi kata kunci berbasis kata benda seperti di halaman
Korea terkadang menemukan kata kunci umum di antara
halaman yang ditautkan. Terlepas dari gagasan yang melekat
tentang kesamaan antar halaman untuk memperkirakan
frekuensi traversal tautan, situasi nol-kesamaan menghalangi
penerapan algoritma PageRank berbobot.
Untuk meningkatkan penerapan PageRank berbobot,
kami mengusulkan metode untuk menjamin beberapa bobot
minimum dan menyesuaikan bobot. PageRank berbobot yang
diusulkan berfungsi sebagai berikut, yang pada dasarnya
berperilaku dengan cara yang sama seperti Qiao algoritme et
al. [3]
Berdasarkan ukuran kemiripan, bobot wij pada node i ke j
dihitung sebagai berikut:

\begin{equation}
    wij = \frac{sij}{\sum. ke Lout(i) sk  } 
\end{equation}

di mana sij adalah kesamaan antara halaman i dan j dan
Lout (i) menunjukkan halaman yang ditunjuk oleh halaman i.
Nilai Peringkat rj dari halaman j diperbarui, hingga semua
peringkat nilai konvergen, sebagai berikut:


%Experiment
\section{Experiments}
isi dari Experiments

%conclusions
\section{conclusions}
isi dari conclusions

%references
\bibliographystyle{IEEEtran}
\bibliography{References}
\end{document}